
\subsection{Cross-section predictions for $t\bar t H$}
\label{sec:results-xsects-tth}
New predictions for the $t\bar t H$ total cross section have recently appeared in Ref.~\cite{Balsach:2025jcw}. These results
combine predictions accurate at NNLO in QCD~\cite{Catani:2022mfv,Devoto:2024nhl} together with soft-gluon resummation up to NNLL~\cite{Kulesza:2015vda,Broggio:2015lya,Broggio:2016lfj,
Kulesza:2017ukk,Kulesza:2018tqz,Broggio:2019ewu,Kulesza:2020nfh}, and include all effects of EW origin up to NLO~\cite{Frixione:2014qaa,Frixione:2015zaa,Frederix:2018nkq}.
Soft-gluon resummation is carried out in two independent framework (SCET and dQCD), which makes it possible to obtain a conservatie estimate of missing higher-order
uncertainties via scale variations. In Ref.~\cite{Balsach:2025jcw}, a thorough estimate of many sources of theoretical uncertainties is carried out.
The only one which may be phenomenologically relevant (besides scale variations, PDF and $\alphas$ uncertainties) is the one stemming 
from the approximation of the two-loop virtual amplitude carried out in order to perform the NNLO computation. Such an uncertainty amounts to
\begin{equation}
    \Delta_{\rm virt} = 0.9\%\,
\end{equation}
for the numbers presented here.
The values for the total cross sections at 13, 13.6 and 14 TeV can be found in Table~\ref{tab:tth-tot}. Cross sections
for the mass values
\begin{equation}
    m_H \in [ 124.6, 125., 125.09, 125.38, 125.6, 126.] \rm GeV\,,
\end{equation}
have been computed. Other values have been obtained via linear interpolation.
For other energies, we advise to employ the numbers
presented in Ref.~\cite{Karlberg:2024zxx}.



\subsection{Cross-section predictions for $t H$}
\label{sec:results-xsects-singleth}
In this section, we provide predictions for $tH$ production. 
For this process, three main production mechanisms concur: 
$t$-channel, $s$-channel and $tWH$ associated production. Their characteristics are very similar to single-top production processes. As it is the case
for these processes, at LHC energies the $t$-channel mode is by far the dominant one. At variance with $ttH$, where the cross section
is proportional to the square of the top-quark Yukawa, for $tH$ it breaks down into three terms: a term independent on $y_t$, a linear term and a quadratic
term. As such, $tH$ production is sensitive to the sign of $y_t$. 

State-of-the-art predictions
for these processes include corrections up to NLO QCD~\cite{Demartin:2015uha,Demartin:2016axk} as well as NLO EW~\cite{Pagani:2020mov}. When
the latter are considered, however, interferences between the three production channels cannot be neglected, and the corresponding breakdown 
therefore becomes ill defined. Thus, the impact of EW corrections can be only assessed
by considering all the channels together (possibly imposing the selection cuts which enhance a given channel). In this case, the effect on the inclusive
cross section has been found in Ref.~\cite{Pagani:2020mov} to be about $-3.5\%$. Given the rather low rate of these processes, together with
the issues related to their inclusion when keeping the three channels separate, EW
corrections are not included in the reference cross sections.\\

All single-top and Higgs cross-sections presented below are computed with {\sc MadGraph5\_aMC@NLO}~\cite{Alwall:2014hca,Frederix:2018nkq}. 

The central value $\mu$ of the renormalisation and factorisation scales is set in a process-dependent manner:
\begin{itemize}
    \item $tH$ ($t$ channel): $\mu= \frac{m_H+m_t}{4}\,$, both for the computation in the 4FS and in the 5FS;
    \item $tH$ ($s$ channel): $\mu= \frac{m_H+m_t}{2}\,$;
    \item $tWH$: $\mu= \frac{m_H+m_t+m_W}{2}\,$;
\end{itemize}
All numbers are computed in the five-flavour scheme. For the $t$-channel production mechanism, quoted scale uncertainties represent the envelope of the
five- and four-flavour scheme computation, both at NLO. In this case, we employ $m_b=4.92 \textrm{GeV}$.\\
For $tWH$, contributions belonging to $t\bar t H$ entering at NLO via real emissions are removed using the so-called ``Diagram removal with interference'' 
scheme (known as DR2), as implemented in {\sc MadGraph5\_aMC@NLO}, see Refs.~\cite{Demartin:2016axk,Frixione:2019fxg}.

Tabs.~\ref{tab:th-t-tot},~\ref{tab:th-t-top} and~~\ref{tab:th-t-atop}
show predictions for $t$-channel production of $tH\,+\,\bar tH$, $tH$ and $\bar tH$ production. Tabs.~\ref{tab:th-s-tot},~\ref{tab:th-s-top} and~~\ref{tab:th-s-atop}
show predictions for $s$-channel production of $tH\,+\,\bar tH$, $tH$ and $\bar tH$ production. Finally, Tab.~\label{tab:thw-tot}
shows predictions for $tW^-H \,+ \,\bar t W^+H$ associated production (the cross section of the two separate processes are equal to each other in this case).
As for $t\bar t H$, cross sections
for the mass values
\begin{equation}
    m_H \in [ 124.6, 125., 125.09, 125.38, 125.6, 126.] \rm GeV\,,
\end{equation}
have been computed. Other values have been obtained via linear interpolation.



