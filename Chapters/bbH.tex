\newcommand{\bbH}{$b\bar{b}H$}
\newcommand{\citere}[1]{Ref.~\cite{#1}}
\newcommand{\citeres}[1]{Refs.~\cite{#1}}
\newcommand{\tab}[1]{Table \ref{#1}}
\newcommand{\nnnlo}{$\text{N}^3\text{LO}$}
\newcommand{\noun}[1]{{\scshape #1}}
\newcommand{\minnlo}{{\noun{MiNNLO$_{\textrm{PS}}$}}}
\newcommand{\nlonnllpart}{NLO+NNLL\textsubscript{part}+$y_by_t$}


In this section, we present a summary of state-of-the-art inclusive predictions for \bbH{} production. Although its production rate at the LHC is comparable to that of $t\bar{t}H$, Higgs boson production in association with bottom quarks poses greater experimental challenges due to the lack of distinct electroweak signatures. The \bbH{} process receives contributions from both top-quark and bottom-quark Yukawa interactions, including interference terms. The small—but non-negligible—value of the bottom-quark mass compared to LHC energies makes the modelling of this process particularly interesting.

In the zero-mass variable-flavour number scheme (ZM-VFNS), the bottom quark can be treated as a massless parton, allowing for the resummation of collinear logarithmic contributions into the parton distribution functions, while neglecting power-suppressed terms in the bottom-quark mass. This approach defines the five-flavour scheme (5FS), where the cross-section is currently known at \nnnlo{} QCD accuracy from~\citere{Duhr:2019kwi}. Additionally, the transverse-momentum and rapidity spectra have been described by resummed analytic predictions in~\citeres{Cal:2023mib,Das:2024pac}. The NNLO predictions~\cite{Harlander:2003ai} have been matched with parton shower at fully-exclusive level in~\citeres{Biello:2024vdh,Gavardi:2025zpf}. These predictions are proportional to the bottom-quark Yukawa contribution, and the interference with top-quark contributions vanishes due to the massless treatment of the bottom-quark.

Alternatively, \bbH{} production can be described within a four-flavour scheme (4FS), where bottom quarks are treated as massive and produced explicitly in the hard scattering together with the Higgs boson. This approach allows for a systematic inclusion of all the bottom-quark mass effects order by order in perturbative QCD and a complete knowledge of the bottom-quark kinematics, beyond the 5FS collinear approximation. Fixed-order predictions in the 4FS are currently known exactly at NLO accuracy, including the interference between top- and bottom-Yukawa induced contributions~\cite{Dittmaier:2003ej,Dawson:2003kb,Wiesemann:2014ioa}. The first determination of NNLO corrections in the bottom-quark Yukawa channel has been presented in~\citere{Biello:2024pgo} using the \minnlo{} method for a parton-shower matching. These corrections have a large positive impact on the cross-section values. The calculation is exact except for the double virtual contribution, which is estimated in the small bottom-mass limit, making use of the analytic knowledge of the two-loop amplitude with massless bottom quarks~\cite{Badger:2021ega,Badger:2024mir}. Beyond the bottom-quark Yukawa contribution, the top-quark loop effects that enter for the first time at NLO are numerically important. Therefore, the NLO prediction for the $y_t^2$ component has been computed separately in the heavy top-quark effective field theory (HEFT) limit~\cite{Deutschmann:2018avk}, highlighting the relevance of this production mode in describing QCD backgrounds for di-Higgs searches~\cite{Manzoni:2023qaf}. In the complete-NLO calculation, additional contributions have been found to compete with the bottom-Yukawa–induced component~\cite{Pagani:2020rsg}.

The two schemes capture different aspects of the heavy-quark mass effects, and a combination of the two approaches is needed for a faithful determination of the cross-section. Matched inclusive predictions have been first performed using a phenomenological combination known as Santander matching~\cite{Harlander:2011aa}. Two systematic approaches have been developed to combine the 4FS and 5FS cross sections while avoiding double-counting: the FONLL method~\cite{Forte:2015hba,Forte:2016sja} and the \nlonnllpart{} approach~\cite{Bonvini:2015pxa,Bonvini:2016fgf}. The FONLL-B variant matches NNLO predictions in the 5FS with NLO predictions in the 4FS and has been shown to agree well with the \nlonnllpart{} results across different values of the Higgs boson mass, as shown in YR4~\cite{LHCHiggsCrossSectionWorkingGroup:2016ypw}. The advantage of the \nlonnllpart{} method is the presence of an explicit dependence on a resummation scale that enables the incorporation of a resummation uncertainty estimation in the predictions. In Tables~\ref{tab:bbH7},~\ref{tab:bbH8},~\ref{tab:bbH13} and~\ref{tab:bbH14} we report cross-section values at respectively 7, 8, 13, and 14 TeV. These numbers are obtained using the following central values for the renormalisation and factorisation scales:
\begin{align}
\mu_{F}=\frac{1}{4}(m_H+2m_b),\hspace{1cm}\mu_R=\frac{1}{2}m_H,.
\end{align}
Theoretical uncertainties from scale variation are estimated via the envelope of the standard 7-point method. In the tables, we also report the dependence on the resummation scale, as well as the parametric uncertainties associated with the parton distribution functions and the strong coupling constant $\alpha_s$.

The predictions at 13.6 TeV are obtained through linear interpolation from cross-section grids at 13 and 14 TeV, and are summarised in~\tab{tab:bbH136}. Tests on the accuracy of the linear interpolation have been performed using existing cross-section predictions at 13.5 TeV. Given the sizeable theoretical perturbative uncertainties in this process, and the good accuracy of the interpolation procedure, dedicated runs at 13.6 TeV are not required. Recently, the \nnnlo{} 5FS prediction has been matched with NLO 4FS cross-sections in~\citere{Duhr:2020kzd} using the FONLL method, reaching a novel level of accuracy in matched fully-inclusive results. Alternative general-mass variable-flavour number scheme (GM-VFNS) approaches, such as that in~\citere{Gauld:2021zmq}, can be employed to obtain matched predictions at the inclusive and fully-differential level.

