Here we describe all input parameters and settings that are common to
all predictions presented in this report. In general we have aimed to use physical parameters from the Review of Particle Physics (PDG)~\cite{ParticleDataGroup:2020ssz}.
\subsection{Fermion masses}

% SJ COMMENT: Gluon fusion input:
\begin{align}
\label{eq:qmasses}
&M_t^\mathrm{OS}=172.5\,\mathrm{GeV},&
&M_b^{\overline{\mathrm{MS}}}(M_b^{\overline{\mathrm{MS}}})=4.18\,\mathrm{GeV},&
&M_c^{\overline{\mathrm{MS}}}(3.0\,\mathrm{GeV})=0.986\,\mathrm{GeV},&
\end{align}
We neglect quark widths. In case the $\overline{\mathrm{MS}}$ scheme is used for the top quark mass, we choose
\begin{align}
M_t^{\overline{\mathrm{MS}}}(M_t^{\overline{\mathrm{MS}}})=162.5\,\mathrm{GeV}.
\end{align}

\subsection{Gauge boson masses}
We use the following on-shell values for the $W$ and $Z$ boson masses
and widths
\begin{align}
        M^\OS_W &= 80.379\GeV\,, \qquad \Gamma^\OS_W=2.085\GeV \notag \\
        M^\OS_Z &= 91.1876\GeV\,, \qquad \Gamma^\OS_W=2.4952\GeV\,.
\end{align}
When needed for EW computations they are translated into their pole
masses~\cite{BARDIN1988539} according to
\begin{align}
        M_V = \frac{M^\OS_V}{\sqrt{1+(\Gamma^\OS_V/M^\OS_V)^2}}\,, \qquad \Gamma_V = \frac{\Gamma^\OS_V}{\sqrt{1+(\Gamma^\OS_V/M^\OS_V)^2}}\,.
\end{align}     
We use the $G_\mu$-scheme~\cite{Denner:2000bj} to compute the fine
structure constant $\alpha$ from $G_F$, $M^\OS_W$, and $M^\OS_Z$
\begin{align}
        \alpha = \frac{\sqrt{2}}{\pi}G_F(M^\OS_W)^2\sin^2\theta_W\,, \quad \sin^2\theta_W = 1-\frac{(M^\OS_W)^2}{(M^\OS_Z)^2}\,,\quad G_F=1.16638\cdot 10^{-5} \GeV^2\,.
\end{align}
This yields a value
\begin{equation}
        \alpha = 0.007565210\,.
\end{equation}


\subsection{PDF and $\alphas$}
Following the PDF4LHC
recommendation~\cite{PDF4LHCWorkingGroup:2022cjn} we use PDF4LHC21\_40
PDF set for all predictions. The value of the strong coupling,
$\alphas$, is given at the $Z$ boson mass
\begin{equation}
        \alphas(M_Z) = 0.1180\pm0.001\,.
\end{equation}  
We estimate the $\alphas$ and PDF uncertainties following the same
recommendation. The $4$-flavour version of the PDF is used whenever a
calculations is performed in the $4$-flavour scheme. The above
combined PDF sets do not contain any photon content. When computing
photon initiate processes we instead use the
LUXqed17\_plus\_PDF4LHC15\_nnlo\_100~\cite{Manohar:2017eqh} set for
the photon \emph{only}. This is not fully consistent as the presence
of the photon in the PDF inevitably impacts the distributions of all
the other partons. However, since the photon-initiated component is
typically very small, this inconsistency is expected to be fully
contained within other theoretical uncertainties.

\subsection{Heavy quark corrections to the parton densities}

% SJ COMMENT: Gluon fusion input:

The heavy quark contributions to the parton distribution functions (PDFs) are known to have different 
scaling violations if compared to the massless case \cite{Blumlein:2016xcy}. In the PDF-fits this has 
to be considered by accounting for the heavy flavor Wilson coefficients to deep-inelastic scattering and 
other hard scattering processes in addition to the light flavor contributions. Furthermore, the heavy 
flavor corrections determine the parton densities in the variable flavor number schemes (VFNS) in 
single heavy 
quark \cite{Buza:1996wv} and two heavy quark cases \cite{Ablinger:2017err}, based on the respective
massive operator matrix elements (OMEs). 
Until very recently these corrections were only available to two-loop order 
\cite{Laenen:1992zk,Laenen:1992xs,Buza:1995ie,Buza:1996wv,Bierenbaum:2007qe,Bierenbaum:2008yu,
Bierenbaum:2009zt,Bierenbaum:2022biv}. For the structure function $F_2(x,Q^2)$ the asymptotic 
description is possible
at the $O(1\%)$ level if $Q^2/m^2_Q \gtrsim 10$, cf. \cite{Buza:1995ie}, being a usual cut in 
PDF-fits 
to avoid higher twist contributions.

A first step to reach the three-loop level consisted in the calculation of a series of three-loop 
Mellin 
moments in Ref.~\cite{Bierenbaum:2009mv}. In the years 2009--2025 all three-loop contributions
to the massive OMEs in the unpolarized and polarized single- and two mass cases have been
calculated. This includes the asymptotic heavy flavor Wilson coefficients for deep inelastic 
scattering. The different contributions were obtained in Refs.~\cite{Ablinger:2010ty,
Ablinger:2014vwa,Ablinger:2014nga,Ablinger:2014lka,Behring:2014eya,
Ablinger:2017err,Ablinger:2018brx,Ablinger:2017xml,
Ablinger:2022wbb,Ablinger:2023ahe,Ablinger:2024xtt,VFNS2,AQG2M}. To perform these 
calculations a lot of new mathematical methods had to be developed, for a survey 
see Ref.~\cite{Blumlein:2023aso}.

One should note that in the singlet case the heavy quark corrections to the structure functions
are large and exhibit quite different scaling violations if compared to the massless case.
Based on these results it is for the first time possible to perform consistent NNLO QCD 
non-singlet \cite{Blumlein:2021lmf}
and singlet analyses with correlated treatment of $\alpha_s, m_c, m_b$ and the 
parton densities now.

Related to it the VFNS scheme is now lifted from two- 
\cite{Ablinger:2017err} to three-loop order, which allows for a refined description 
of the process of a heavy quark becoming light at large virtualities.

\paragraph{Contributors:} J.~Bl\"{u}mlein
