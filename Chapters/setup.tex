Here we describe all input parameters and settings that are common to
all predictions presented in this report. In general we have aimed to use physical parameters from the Review of Particle Physics (PDG)~\cite{ParticleDataGroup:2020ssz}.
\subsection{Fermion masses}

\subsection{Gauge boson masses}
We use the following on-shell values for the $W$ and $Z$ boson masses
and widths
\begin{align}
        M^\OS_W &= 80.379\GeV\,, \qquad \Gamma^\OS_W=2.085\GeV \notag \\
        M^\OS_Z &= 91.1876\GeV\,, \qquad \Gamma^\OS_W=2.4952\GeV\,.
\end{align}
When needed for EW computations they are translated into their pole
masses~\cite{BARDIN1988539} according to
\begin{align}
        M_V = \frac{M^\OS_V}{\sqrt{1+(\Gamma^\OS_V/M^\OS_V)^2}}\,, \qquad \Gamma_V = \frac{\Gamma^\OS_V}{\sqrt{1+(\Gamma^\OS_V/M^\OS_V)^2}}\,.
\end{align}     
We use the $G_\mu$-scheme~\cite{Denner:2000bj} to compute the fine
structure constant $\alpha$ from $G_F$, $M^\OS_W$, and $M^\OS_Z$
\begin{align}
        \alpha = \frac{\sqrt{2}}{\pi}G_F(M^\OS_W)^2\sin^2\theta_W\,, \quad \sin^2\theta_W = 1-\frac{(M^\OS_W)^2}{(M^\OS_Z)^2}\,,\quad G_F=1.16638\cdot 10^{-5} \GeV^2\,.
\end{align}
This yields a value
\begin{equation}
        \alpha = 0.007565210\,.
\end{equation}


\subsection{PDF and $\alphas$}
Following the PDF4LHC
recommendation~\cite{PDF4LHCWorkingGroup:2022cjn} we use PDF4LHC21\_40
PDF set for all predictions. The value of the strong coupling,
$\alphas$, is given at the $Z$ boson mass
\begin{equation}
        \alphas(M_Z) = 0.1180\pm0.001\,.
\end{equation}  
We estimate the $\alphas$ and PDF uncertainties following the same
recommendation. The $4$-flavour version of the PDF is used whenever a
calculations is performed in the $4$-flavour scheme. The above
combined PDF sets do not contain any photon content. When computing
photon initiate processes we instead use the
LUXqed17\_plus\_PDF4LHC15\_nnlo\_100~\cite{Manohar:2017eqh} set for
the photon \emph{only}. This is not fully consistent as the presence
of the photon in the PDF inevitably impacts the distributions of all
the other partons. However, since the photon-initiated component is
typically very small, this inconsistency is expected to be fully
contained within other theoretical uncertainties.
