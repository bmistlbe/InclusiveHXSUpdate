
\paragraph{Short summary of theory work} In this section, we shortly summarise theory work which has been carried out over the years to describe Higgs production via VBF at the LHC.
Recently, a comparative study at the differential level at fixed order and including parton-shower corrections has been made public~\cite{TOBEADDED}.
In addition to also provide recommendations for theory uncertainties in parton-shower simulations, it contains a pedagogical description of higher-order corrections in VBF as well as a review of the state of the art.
We therefore refer the interested reader to Ref.~\cite{TOBEADDED}.
In addition, regarding fixed-order calculations, the Les Houches wishlist~\cite{LesHouches} also provides a useful resource for recent developments.
For the sake of completeness, we nonetheless provide a short review of recent and state-of-the-art theory calculations for VBF at the LHC.

The electroweak (EW) production of a Higgs boson in association with two jets features several contributions, the purely VBF one and the VH one.
To that end, the quality of the VBF approximation has been studied in the literature along with the impact of higher-multiplicity contributions~\cite{Figy:2003nv,Berger:2004pca,Ciccolini:2007jr,Bolzoni:2011cu,Campanario:2013fsa,Campanario:2018ppz,TOBEADDED}.

The current state of the art for EW corrections is NLO EW ~\cite{Ciccolini:2007jr,Ciccolini:2007ec} and these are used in the present case.
At the differential level, the state of the art for QCD corrections is
NNLO QCD in the VBF approximation~\cite{Cacciari:2015jma,Cruz-Martinez:2018rod} while for inclusive numbers at the in the present case, N3LO QCD \cite{Dreyer:2016oyx}.
Given that state-of-the-art QCD computations are done in the VBF approximation which neglect the cross talk between the two quark lines, the impact of non-factorisable corrections has been studies in details in the literature~\cite{Liu:2019tuy,Dreyer:2020urf,Asteriadis:2023nyl,Bronnum-Hansen:2023vzh,Long:2023mvc,Gates:2023iiv}.
In the present predictions, non-factorisable corrections are included though only as an uncertainty.
As mentioned above, the EW production of Hjj can happen through several production mechanisms but in addition, other contributions at different orders in perturbative theory also lead to the same final state~\cite{Andersen:2006ag,Andersen:2007mp,Harlander:2008xn,Greiner:2016awe,Greiner:2015jha,Andersen:2017kfc,Andersen:2018tnm,Andersen:2018kjg,Chen:2021azt,Andersen:2022zte}.
A summary of these irreducible backgrounds can be found in Ref.~\cite{TOBEADDED}.
Also, the characteristics of the VBF production mechanism have been study in several phase-space regions, including the ones where the Higgs boson is boosted~\cite{Becker:2020rjp,Buckley:2021gfw}.
Finally, while historically the focus of both theory and experimental work has been on stable Higgs, recent works~\cite{Asteriadis:2021gpd,Asteriadis:2024nbg} have shown that considering the decay products of the Higgs boson long with its production mechanism can induce non-trivial effects.

In addition to fixed-order corrections, significant work related to parton shower has been done over the years.
Parton-shower corrections matched to QCD~\cite{Nason:2009ai,Frixione:2013mta} and even EW~\cite{Jager:2022acp} corrections are available at NLO accuracy.
Also, multi-jet merged results have been investigated over the years~\cite{Hoche:2021mkv,Chen:2021phj}.
Uncertainties related to NLO QCD+PS simulations have been also studied in details ~\cite{Jager:2020hkz,Buckley:2021gfw,Hoche:2021mkv}.
Recommendations regarding such uncertainties have been made in Ref.~\cite{TOBEADDED}.
Finally, soft-physics effects such as multi-parton interaction and hadronisation~\cite{Hoche:2021mkv,Bittrich:2021ztq} have been studied.
Developments regarding next-to-leading-logarithm showers for VBF~\cite{vanBeekveld:2023chs} have also been made.

\paragraph{Inclusive numbers}

The results compiled in Tables~\ref{tab:vbf_XStot_7}-\ref{tab:vbf_XStot_14} are combined according to
\begin{align}
  \label{eq:vbf-xsec-combined}
  \sigma^{\VBF} = \sigma_{\NNNLO}^{\DIS} (1+\delta_{\ELWK}) + \sigma_{\gamma} ,
\end{align}
%
where $\sigma_{\NNNLO}^{\DIS}$ is the $\NNNLO$ QCD cross section in the DIS approximation.
The relative EW corrections omitting the photon-induced contributions are denoted by $\delta_{\ELWK}$ while the photon-induced contributions are enclosed in $\sigma_{\gamma}$.
On the other hand, the theory uncertainties are computed as
\begin{align}
  \label{eq:vbf-TU}
  \Delta_{\mathrm{TU}} = \max\left\{0.5\%,\delta_{\ELWK}^2\right\} + \frac{|\sigma_{\mbox{\scriptsize nf}}| + |\sigma_{\mbox{\scriptsize s/t/u}}|}{\sigma^{\VBF}}\%
\end{align}
for $\sqrt{s}=\{13,13.6,14\}$ TeV. For the legacy numbers corresponding to $\sqrt{s}=\{7,8\}$ TeV the non-factorisable contribution, $\sigma_{\mbox{\scriptsize nf}}$, was not computed, and we instead set
\begin{align}
  \label{eq:vbf-TU}
  \Delta_{\mathrm{TU}} = \max\left[\max\left\{0.5\%,\delta_{\ELWK}^2\right\} + \frac{|\sigma_{\mbox{\scriptsize s/t/u}}|}{\sigma^{\VBF}}\%,1.0\%\right].
\end{align}
In fact, in this case it always corresponds to $1\%$.

\paragraph{Contributors}

The $\NNNLO$ cross section has been provided by Alexander Karlberg based on Ref.~\cite{Dreyer:2016oyx}.
The NLO EW corrections, including the photon-induced contributions have been provided by Alexander M\"uck based on Ref.~\cite{Denner:2014cla}.
The size of the interferences between $s/t/u$ contributions have been estimated with the help of {\sc Hawk}~\cite{Denner:2014cla} by Alexander Mück.
The non-factorisable corrections have been provided by Christian Br{\o}nnum-Hansen based on Ref.~\cite{????}.
